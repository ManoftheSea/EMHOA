\documentclass[10pt, letterpaper]{article}
\usepackage{enumitem}
\usepackage[letterpaper, margin=.5in]{geometry}
\usepackage{hyperref}
\usepackage[utf8]{inputenc}
\usepackage{textcomp}

\title{ENGLISH MEADOWS HOMEOWNERS' ASSOCIATION,\linebreak[0] Architectural Review Committee Policy}
\author{EMHOA \thanks{digitized by Derek LaHousse}}
\date{\today}

\setlist[enumerate]{itemsep=0mm}
\begin{document}
\maketitle

As Adopted By Resolution of the Board of Directors on 19 January 1995.

\section{PURPOSE}
\begin{quote}
  ``To enhance and protect the value, attractiveness and desirability of the lots.''
  \hfill (Protective~Covenants,~Conditions,~and Restrictions~[C\&R].)
\end{quote}
Using a community perspective provide reasonable guidelines and consistency for those items covered by the covenants WITHOUT dictating taste or design.
The intended direction to the A.R.C. is to maintain ``Curb Appeal'' and not to infringe unnecessarily on homeowner's rights to do as they please with their back yards.

\section{AUTHORITY}
\begin{quote}
  ``The Board of Directors shall have the power to enforce the covenants, restrictions, easements, conditions, agreements, and liens provided in the Declaration''
  \hfill (By-Laws,~Article~VII,~Sect.~1(b).)
\end{quote}
\begin{quote}
  ``No building shall be erected, placed, or altered on any building lot in this subdivision until the building plans, specifications, and plot plans showing the location of such building have been approved in writing as to the conformity and harmony of external design with existing structures in the subdivision and as to the location of the building with respect to topography and finished ground elevation by the Architectural Committee to be selected by [the BOD].
  All such approvals shall be determined by said Architectural Committee, in its sole discretion, and no member of the Association, jointly or in concert, shall have any recourse from any determination of said Architectural Committee.''
  \hfill (C\&R Sect. 8, Para. 10)
\end{quote}
\begin{quote}
  ``No fence, hedge, wall, or other dividing instrumentality shall be constructed or maintained on any lot except upon approval of the Architectural Committee...''
  \hfill (C\&R Sect. 8, Para. 8)
\end{quote}

\section{SCOPE} The following are examples of what must be submitted for approval per the above authority:
\begin{enumerate}
  \item Sheds of any type, permanent or ``temporary''.
  \item Decks, Covered patios, Porches, Additions.
  \item Garages, or covered Carports.
  \item Fences of any style that divide a lot or create a distinctly separate area including fenced-in dog runs.
  \item Roof-Mounted Solar Panels.
  \item Any other Alterations to any existing or subsequently approved buildings.
\end{enumerate}

Note: Most of these examples require appropriate County permits.
We suggest Association approval comes first before spending money on the permit process.

Following are examples that DO NOT need approval per our existing Covenants \& Restrictions:
\begin{enumerate}
  \item Pools - Above/Below Ground. (Requires County Approval)
  \item Driveways.
  \item Uncovered Patios, Walkways unless altering a building.
  \item Play-sets, Basketball Hoops.
  \item Landscaping.
  \item Retaining Walls such as along a driveway or at the foundation.
  \item Satellite Dishes. (Requires County Approval)
  \item Fish Ponds, Water Gardens.
  \item Exterior Lighting.
  \item Landscaping Fences such as at property corners.
\end{enumerate}

\section{ADMINISTRATION}
\begin{enumerate}
  \item The BOD shall appoint the members of the Architectural Committee.
    \hfill (Bylaws, Article X)
  \item The President shall appoint the Chairman of the committee.
    \hfill (Bylaws, Article VIII, Sect. 8(a).)
  \item Any Covenant \& Restriction may be amended if approved by three quarters (\textthreequarters) of the Association.
    \hfill (C\&R, Sect. IX, 3.)
  \item The Architectural Committee will consist of five (5) regular members. Any number of alternates may be designated by the BOD. A geographical spread of members is highly desired.
  \item One (1) regular member must also be a Director.
  \item A reasonable attempt must be made by the Chairperson to notify all regular committee members of a meeting.
    If a regular member indicates s/he cannot make a meeting then any alternate member may be called in an attempt to get five (5) members present at a meeting.
  \item A quorum is three (3) members present at a meeting.
  \item For any amount of committee members present at a meeting must have at least three (3) like votes to either approve or disapprove the submitted plans.
  \item In the event three (3) like votes are not cast (such as a 2-2 vote with four members present or a 2-1 vote with only three members present) another meeting will be convened by the chairperson to include the absent regular members and, if necessary, any alternate members.
    A member's original vote will stand and be recorded if s/he is not present at this second meeting.
  \item All Committee members present and voting must have their names annotated in the signature block of Reply Letter.
    Only the Chairperson or Acting Chairperson needs to sign.
  \item Send/Deliver original Reply Letter to the homeowner.
    Include the statement:
    ``Approval is valid for one year from the date of this letter. If activity is not started by then, re-approval is required.''
    If a submission is disapproved include a detailed reason in the Reply Letter and indicate what would be required for approval.
  \item Forward submitted documents and a copy of the Committee Reply Letter to the President EMHOA for permanent filing in the Association Records.
  \item The Committee Chairperson will keep a copy of at least the Reply Letter for committee members' review at the next meeting.
  \item The Chairperson is responsible for monitoring a project for compliance with what was approved.
    If found to be in non-compliance the President will be notified who will discuss the matter with the Board of Directors before any action is contemplated.
\end{enumerate}

\section{TIMELINE}
\begin{enumerate}
  \item The Chairperson will call a meeting within twenty (20) days of receiving submitted documents from a homeowner.
  \item Earlier consideration if requested due to extenuating circumstances will be accommodated, if possible.
  \item The homeowner shall receive notification within ten (10) days of the meeting.
  \item If the project is not commenced within one (1) year of the approval, re-submission for approval is required.
  \item Where natural screening is specified in the approval a reasonable but definite timeline will be set for compliance.
\end{enumerate}

\section{HANDLING COMPLAINTS}
\begin{enumerate}
  \item ANY Homeowner, whether Director, Officer, Committee Member or not may take action by themselves or through the Association if a breach of the C\&R's occurs.
    \hfill (C\&R, Sect. IX, 1.)
  \item If a neighbor to neighbor talk does not resolve the issue you can request the Association to consider action by submitting a signed, dated letter to any Officer or Director.
    The full Board will then convene to discuss the Association's response.
    A ``heads up'' phone call is optional and confidentiality will be maintained, as much as is practical.
\end{enumerate}

Send Plans, Sketches, Plot Location, Brochures, Photos, Descriptions as Appropriate to the Architectural Review Committee Chairperson or the President, EMHOA for Approval.
\end{document}
